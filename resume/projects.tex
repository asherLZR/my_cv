\cvsection{Projects}
\begin{cventries}
\cventry
    {Hackamon 2019, Second Place Winners}
    {Perspective}
    {Victoria, Australia}
    {Apr. 2019}
    {
      \begin{cvitems}
        \item {While it is well-known that the best way to learn is through teaching, this is a severely underutilised mode of learning. We addressed this by proposing a video learning website, made with Node.js and Vue.js, where the students are the content providers.}
        \item {Complementing traditional peer-to-peer learning methods like tutorials and forum posts, Perspective was online and encouraged teaching as another form of validating understanding of content. Such a contract benefits both uploaders and viewers. At its core, Perspective was cross-disciplinary as such videos may be extended to viewers accross all faculties.}
      \end{cvitems}
    }
\cventry
    {Personal Project}
    {Z-Algorithm Visualised}
    {Victoria, Australia}
    {Apr. 2019}
    {
      \begin{cvitems}
        \item {To better understand web technologies (especially Vue.js), created a {\color{red} \href{https://zalg.herokuapp.com}{simple visualisation}} of Gusfield's Z-Algorithm.}
      \end{cvitems}
    }
  \cventry
    {Unihack 2019}
    {Meta}
    {Victoria, Australia}
    {Mar. 2019}
    {
      \begin{cvitems}
        \item {To address the siloing of content by major online news curators, built a news archival website which performed NLP and analytics on user-read articles collected by a browser extension. The website was envisioned to be used as a reflective tool for users to engage with their personal reading history, outside the influence of recommendation engines.}
        \item {Built in 24hrs with a team of 5. It was my first experience leading a tech team and building a live website from scratch.}
        \item {In a crash course, learnt technologies such as MongoDB, Heroku, React, and Express.js.}
      \end{cvitems}
    }
\end{cventries}
